\documentclass[fontsize=12pt, a4paper, headinclude, twoside=false, parskip=half+, pagesize=auto, numbers=noenddot, plainheadsepline, open=right, toc=listof, toc=bibliography, chapteratlists=0pt]{scrreprt}

%\usepackage{calc}
%\usepackage[a4paper,textwidth=\textwidth]{geometry}
%\addtolength{\paperwidth}{6cm}
%\addtolength{\marginparwidth}{6cm}


% Allgemeines
\usepackage[automark]{scrpage2} % Kopf- und Fußzeilen
\usepackage{amsmath,marvosym} % Mathesachen
\usepackage[T1]{fontenc} % Ligaturen, richtige Umlaute im PDF
\usepackage[utf8]{inputenc}% UTF8-Kodierung für Umlaute usw
%\usepackage{perpage} %the perpage package
%\MakePerPage{footnote} %the perpage package command
\usepackage{caption}
\usepackage[Q=yes]{examplep}
\usepackage[colorinlistoftodos]{todonotes}

% Schriften
\usepackage{setspace} % Zeilenabstand
\onehalfspacing % 1,5 Zeilen
\usepackage{lmodern}

% Schriften-Größen
\setkomafont{chapter}{\Huge\rmfamily} % Überschrift der Ebene
\setkomafont{section}{\Large\rmfamily}
\setkomafont{subsection}{\large\rmfamily}
\setkomafont{subsubsection}{\small\rmfamily}
\setkomafont{chapterentry}{\large\rmfamily} % Überschrift der Ebene in Inhaltsverzeichnis
\setkomafont{descriptionlabel}{\bfseries\rmfamily} % für description Umgebungen
\setkomafont{captionlabel}{\small\bfseries}
\setkomafont{caption}{\small}

% Custom colors
\definecolor{deepblue}{rgb}{0,0,0.5}
\definecolor{deepred}{rgb}{0.6,0,0}
\definecolor{deepgreen}{rgb}{0,0.5,0}
\definecolor{grau}{rgb}{0.5,0.5,0.5}

% Sprache: Deutsch
\usepackage[ngerman]{babel} % Silbentrennung

% PDF
\usepackage[ngerman, breaklinks=true]{hyperref}
\usepackage[final]{microtype} % mikrotypographische Optimierungen
\usepackage{url}
\usepackage{pdflscape} % einzelne Seiten drehen können

% Tabellen
\usepackage{multirow} % Tabellen-Zellen über mehrere Zeilen
\usepackage{multicol} % mehre Spalten auf eine Seite
\usepackage{tabularx} % Für Tabellen mit vorgegeben Größen
\usepackage{longtable} % Tabellen über mehrere Seiten
\usepackage{array}
\usepackage{float}
\usepackage{booktabs}

% Diagramme
%\usepackage{tikz}
%\usepackage{pgfplotstable}
%\usepackage{pgfplots}
%\usetikzlibrary{trees}

%  Bibliographie
\usepackage{bibgerm} % Umlaute in BibTeX
\usepackage{cite}
%% For changing the IEEE citation style
\makeatletter
\def\bstctlcite{\@ifnextchar[{\@bstctlcite}{\@bstctlcite[@auxout]}}
\def\@bstctlcite[#1]#2{\@bsphack
	\@for\@citeb:=#2\do{%
		\edef\@citeb{\expandafter\@firstofone\@citeb}%
		\if@filesw\immediate\write\csname #1\endcsname{\string\citation{\@citeb}}\fi}%
	\@esphack}
\makeatother


% Bilder
\usepackage{graphicx} % Bilder
\graphicspath{{images/}}
\DeclareGraphicsExtensions{.pdf,.png,.jpg} % bevorzuge pdf-Dateien
\usepackage[all]{hypcap} % Beim Klicken auf Links zum Bild und nicht zu Caption gehen


% Bildunterschrift
\usepackage{caption}
\usepackage{chngcntr}
\counterwithout{figure}{chapter}
\setcapindent{0em} % kein Einrücken der Caption von Figures und Tabellen
\setcapwidth[c]{0.9\textwidth}
\setlength{\abovecaptionskip}{0.2cm} % Abstand der zwischen Bild- und Bildunterschrift

% Quellcode
\usepackage{listings} % für Formatierung in Quelltexten

% Default fixed font does not support bold face
\DeclareFixedFont{\ttb}{T1}{txtt}{bx}{n}{10} % for bold
\DeclareFixedFont{\ttm}{T1}{txtt}{m}{n}{10}  % for normal

\lstset{
    extendedchars=true,
	numberbychapter=true,
	basicstyle=\scriptsize\ttm,
 	frame=l,
	captionpos=b,		
	xleftmargin=5pt,
	tabsize=2,
    numbers=none,
	numberstyle=\scriptsize,
	numbersep=10pt,
	breakautoindent  = true,
	breaklines       = true,
	breakatwhitespace = true,
	prebreak=\raisebox{0ex}[0ex][0ex]{\ensuremath{\hookleftarrow}}
}


% Python environment
\lstnewenvironment{python}[2][]
{
\vspace{10pt}
\lstset{
    caption=#1,
    label=lst.#2,
	language=Python,
	commentstyle=\color{grau},
	morekeywords={self}, % Add keywords here
	deletekeywords=[2]{help}, 
	keywordstyle=\ttb\color{deepblue},
	stringstyle=\color{deepgreen},
	showstringspaces=false,
	}
}
{}


% typearea berechnet einen sinnvollen Satzspiegel (das heißt die Seitenränder) siehe auch http://www.ctan.org/pkg/typearea. Diese Berechnung befindet sich am Schluss, damit die Einstellungen oben berücksichtigt werden
%\typearea{14} 



% Eigene Befehle %%%%%%%%%%%%%%%%%%%%%%%%%%%%%%%%%%%%%%%%%%%%%%%%%
\newcommand{\image}[4][!h]{
	\begin{figure}[#1]
		\centering
		\vspace{1ex}
		\includegraphics[#3]{images/#2}
		\caption[#4]{#4}\label{img.#2}      
		\vspace{1ex}
	\end{figure}
}

